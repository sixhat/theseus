% Generated by Sphinx.
\def\sphinxdocclass{report}
\documentclass[letterpaper,10pt,english]{sphinxmanual}
\usepackage[utf8]{inputenc}
\DeclareUnicodeCharacter{00A0}{\nobreakspace}
\usepackage[T1]{fontenc}
\usepackage{babel}
\usepackage{times}
\usepackage[Bjarne]{fncychap}
\usepackage{longtable}
\usepackage{sphinx}


\title{Theseus Documentation}
\date{October 21, 2010}
\release{0.6.0rc}
\author{David Rodrigues}
\newcommand{\sphinxlogo}{}
\renewcommand{\releasename}{Release}
\makeindex

\makeatletter
\def\PYG@reset{\let\PYG@it=\relax \let\PYG@bf=\relax%
    \let\PYG@ul=\relax \let\PYG@tc=\relax%
    \let\PYG@bc=\relax \let\PYG@ff=\relax}
\def\PYG@tok#1{\csname PYG@tok@#1\endcsname}
\def\PYG@toks#1+{\ifx\relax#1\empty\else%
    \PYG@tok{#1}\expandafter\PYG@toks\fi}
\def\PYG@do#1{\PYG@bc{\PYG@tc{\PYG@ul{%
    \PYG@it{\PYG@bf{\PYG@ff{#1}}}}}}}
\def\PYG#1#2{\PYG@reset\PYG@toks#1+\relax+\PYG@do{#2}}

\def\PYG@tok@gu{\let\PYG@bf=\textbf\def\PYG@tc##1{\textcolor[rgb]{0.50,0.00,0.50}{##1}}}
\def\PYG@tok@gt{\def\PYG@tc##1{\textcolor[rgb]{0.00,0.25,0.82}{##1}}}
\def\PYG@tok@gs{\let\PYG@bf=\textbf}
\def\PYG@tok@gr{\def\PYG@tc##1{\textcolor[rgb]{1.00,0.00,0.00}{##1}}}
\def\PYG@tok@cm{\let\PYG@it=\textit\def\PYG@tc##1{\textcolor[rgb]{0.25,0.50,0.56}{##1}}}
\def\PYG@tok@vg{\def\PYG@tc##1{\textcolor[rgb]{0.73,0.38,0.84}{##1}}}
\def\PYG@tok@m{\def\PYG@tc##1{\textcolor[rgb]{0.13,0.50,0.31}{##1}}}
\def\PYG@tok@mh{\def\PYG@tc##1{\textcolor[rgb]{0.13,0.50,0.31}{##1}}}
\def\PYG@tok@go{\def\PYG@tc##1{\textcolor[rgb]{0.19,0.19,0.19}{##1}}}
\def\PYG@tok@ge{\let\PYG@it=\textit}
\def\PYG@tok@gd{\def\PYG@tc##1{\textcolor[rgb]{0.63,0.00,0.00}{##1}}}
\def\PYG@tok@il{\def\PYG@tc##1{\textcolor[rgb]{0.13,0.50,0.31}{##1}}}
\def\PYG@tok@cs{\def\PYG@tc##1{\textcolor[rgb]{0.25,0.50,0.56}{##1}}\def\PYG@bc##1{\colorbox[rgb]{1.00,0.94,0.94}{##1}}}
\def\PYG@tok@cp{\def\PYG@tc##1{\textcolor[rgb]{0.00,0.44,0.13}{##1}}}
\def\PYG@tok@gi{\def\PYG@tc##1{\textcolor[rgb]{0.00,0.63,0.00}{##1}}}
\def\PYG@tok@gh{\let\PYG@bf=\textbf\def\PYG@tc##1{\textcolor[rgb]{0.00,0.00,0.50}{##1}}}
\def\PYG@tok@ni{\let\PYG@bf=\textbf\def\PYG@tc##1{\textcolor[rgb]{0.84,0.33,0.22}{##1}}}
\def\PYG@tok@nl{\let\PYG@bf=\textbf\def\PYG@tc##1{\textcolor[rgb]{0.00,0.13,0.44}{##1}}}
\def\PYG@tok@nn{\let\PYG@bf=\textbf\def\PYG@tc##1{\textcolor[rgb]{0.05,0.52,0.71}{##1}}}
\def\PYG@tok@no{\def\PYG@tc##1{\textcolor[rgb]{0.38,0.68,0.84}{##1}}}
\def\PYG@tok@na{\def\PYG@tc##1{\textcolor[rgb]{0.25,0.44,0.63}{##1}}}
\def\PYG@tok@nb{\def\PYG@tc##1{\textcolor[rgb]{0.00,0.44,0.13}{##1}}}
\def\PYG@tok@nc{\let\PYG@bf=\textbf\def\PYG@tc##1{\textcolor[rgb]{0.05,0.52,0.71}{##1}}}
\def\PYG@tok@nd{\let\PYG@bf=\textbf\def\PYG@tc##1{\textcolor[rgb]{0.33,0.33,0.33}{##1}}}
\def\PYG@tok@ne{\def\PYG@tc##1{\textcolor[rgb]{0.00,0.44,0.13}{##1}}}
\def\PYG@tok@nf{\def\PYG@tc##1{\textcolor[rgb]{0.02,0.16,0.49}{##1}}}
\def\PYG@tok@si{\let\PYG@it=\textit\def\PYG@tc##1{\textcolor[rgb]{0.44,0.63,0.82}{##1}}}
\def\PYG@tok@s2{\def\PYG@tc##1{\textcolor[rgb]{0.25,0.44,0.63}{##1}}}
\def\PYG@tok@vi{\def\PYG@tc##1{\textcolor[rgb]{0.73,0.38,0.84}{##1}}}
\def\PYG@tok@nt{\let\PYG@bf=\textbf\def\PYG@tc##1{\textcolor[rgb]{0.02,0.16,0.45}{##1}}}
\def\PYG@tok@nv{\def\PYG@tc##1{\textcolor[rgb]{0.73,0.38,0.84}{##1}}}
\def\PYG@tok@s1{\def\PYG@tc##1{\textcolor[rgb]{0.25,0.44,0.63}{##1}}}
\def\PYG@tok@vc{\def\PYG@tc##1{\textcolor[rgb]{0.73,0.38,0.84}{##1}}}
\def\PYG@tok@sh{\def\PYG@tc##1{\textcolor[rgb]{0.25,0.44,0.63}{##1}}}
\def\PYG@tok@ow{\let\PYG@bf=\textbf\def\PYG@tc##1{\textcolor[rgb]{0.00,0.44,0.13}{##1}}}
\def\PYG@tok@mf{\def\PYG@tc##1{\textcolor[rgb]{0.13,0.50,0.31}{##1}}}
\def\PYG@tok@bp{\def\PYG@tc##1{\textcolor[rgb]{0.00,0.44,0.13}{##1}}}
\def\PYG@tok@c1{\let\PYG@it=\textit\def\PYG@tc##1{\textcolor[rgb]{0.25,0.50,0.56}{##1}}}
\def\PYG@tok@kc{\let\PYG@bf=\textbf\def\PYG@tc##1{\textcolor[rgb]{0.00,0.44,0.13}{##1}}}
\def\PYG@tok@c{\let\PYG@it=\textit\def\PYG@tc##1{\textcolor[rgb]{0.25,0.50,0.56}{##1}}}
\def\PYG@tok@sx{\def\PYG@tc##1{\textcolor[rgb]{0.78,0.36,0.04}{##1}}}
\def\PYG@tok@err{\def\PYG@bc##1{\fcolorbox[rgb]{1.00,0.00,0.00}{1,1,1}{##1}}}
\def\PYG@tok@kd{\let\PYG@bf=\textbf\def\PYG@tc##1{\textcolor[rgb]{0.00,0.44,0.13}{##1}}}
\def\PYG@tok@ss{\def\PYG@tc##1{\textcolor[rgb]{0.32,0.47,0.09}{##1}}}
\def\PYG@tok@sr{\def\PYG@tc##1{\textcolor[rgb]{0.14,0.33,0.53}{##1}}}
\def\PYG@tok@mo{\def\PYG@tc##1{\textcolor[rgb]{0.13,0.50,0.31}{##1}}}
\def\PYG@tok@kn{\let\PYG@bf=\textbf\def\PYG@tc##1{\textcolor[rgb]{0.00,0.44,0.13}{##1}}}
\def\PYG@tok@mi{\def\PYG@tc##1{\textcolor[rgb]{0.13,0.50,0.31}{##1}}}
\def\PYG@tok@gp{\let\PYG@bf=\textbf\def\PYG@tc##1{\textcolor[rgb]{0.78,0.36,0.04}{##1}}}
\def\PYG@tok@o{\def\PYG@tc##1{\textcolor[rgb]{0.40,0.40,0.40}{##1}}}
\def\PYG@tok@kr{\let\PYG@bf=\textbf\def\PYG@tc##1{\textcolor[rgb]{0.00,0.44,0.13}{##1}}}
\def\PYG@tok@s{\def\PYG@tc##1{\textcolor[rgb]{0.25,0.44,0.63}{##1}}}
\def\PYG@tok@kp{\def\PYG@tc##1{\textcolor[rgb]{0.00,0.44,0.13}{##1}}}
\def\PYG@tok@w{\def\PYG@tc##1{\textcolor[rgb]{0.73,0.73,0.73}{##1}}}
\def\PYG@tok@kt{\def\PYG@tc##1{\textcolor[rgb]{0.56,0.13,0.00}{##1}}}
\def\PYG@tok@sc{\def\PYG@tc##1{\textcolor[rgb]{0.25,0.44,0.63}{##1}}}
\def\PYG@tok@sb{\def\PYG@tc##1{\textcolor[rgb]{0.25,0.44,0.63}{##1}}}
\def\PYG@tok@k{\let\PYG@bf=\textbf\def\PYG@tc##1{\textcolor[rgb]{0.00,0.44,0.13}{##1}}}
\def\PYG@tok@se{\let\PYG@bf=\textbf\def\PYG@tc##1{\textcolor[rgb]{0.25,0.44,0.63}{##1}}}
\def\PYG@tok@sd{\let\PYG@it=\textit\def\PYG@tc##1{\textcolor[rgb]{0.25,0.44,0.63}{##1}}}

\def\PYGZbs{\char`\\}
\def\PYGZus{\char`\_}
\def\PYGZob{\char`\{}
\def\PYGZcb{\char`\}}
\def\PYGZca{\char`\^}
% for compatibility with earlier versions
\def\PYGZat{@}
\def\PYGZlb{[}
\def\PYGZrb{]}
\makeatother

\begin{document}

\maketitle
\tableofcontents
\phantomsection\label{index::doc}


Latest release: 0.6.0rc

Latest update: October 21, 2010


\chapter{Contents}
\label{index:contents}\label{index:theseus-s-documentation}

\section{What is Theseus?}
\label{whatis:what-is-theseus}\label{whatis::doc}
\emph{A brief overview on what Theseus is, and is not}

It's a collection of scripts, programs and libraries

mainly consist of bash scripts + python scripts

It his the responsible for collecting and processing newspaper pages for the observatorium


\section{What will Theseus be at version 1.0?}
\label{whatis:what-will-theseus-be-at-version-1-0}
Theseus will end up (eventually) being made of 3 groups of programs/scripts:
\begin{itemize}
\item {} 
\textbf{Crawler} (for online gathering of news items)

\item {} 
\textbf{Processor} (for processing of textual data)

\item {} 
\textbf{Utils} (acessory methods and utilities for pre and post processing)

\end{itemize}


\section{Theseus now!}
\label{theseus:theseus-now}\label{theseus::doc}
The Theseus Project is a collection of several scripts that help the scientist to manipulate text documents in manner to extract useful information.

Theseus is part of \href{http://theobservatorium.eu}{http://theobservatorium.eu} project.

\textbf{This is the main module for data processing. It's where several classes that old the data are defined.}
\phantomsection\label{theseus:module-theseus}\index{theseus (module)}

\subsection{theseus.py}
\label{theseus:theseus-py}
A Python Library for text processing in The Observatorium project

\href{http://theobservatorium.eu/}{http://theobservatorium.eu/}

Created by David Rodrigues on 2010-02-03.

Copyright (c) 2010 Sixhat Pirate Parts. All rights reserved.
\index{Channel (class in theseus)}

\begin{fulllineitems}
\phantomsection\label{theseus:theseus.Channel}\pysiglinewithargsret{\strong{class }\code{theseus.}\bfcode{Channel}}{\emph{label}}{}
A Channel contains all documents of a certain channel
\begin{itemize}
\item {} 
\emph{label} is a string

\item {} 
\emph{doc} is a DocNode

\end{itemize}

\end{fulllineitems}

\index{DocNode (class in theseus)}

\begin{fulllineitems}
\phantomsection\label{theseus:theseus.DocNode}\pysiglinewithargsret{\strong{class }\code{theseus.}\bfcode{DocNode}}{\emph{idn='`}, \emph{fnm='`}, \emph{txt='`}, \emph{ttl='`}, \emph{lang='en'}}{}
The DocNode is the basic structue that olds each document in a corpus
\begin{itemize}
\item {} 
\emph{idn} Id number of the node

\item {} 
\emph{fnm} File name of the Document

\item {} 
\emph{txt} Text of the Document

\item {} 
\emph{ttl} Time to Live

\item {} 
\emph{lang='en'} The language of the text, defaults to english

\end{itemize}
\index{extractSentences() (theseus.DocNode method)}

\begin{fulllineitems}
\phantomsection\label{theseus:theseus.DocNode.extractSentences}\pysiglinewithargsret{\bfcode{extractSentences}}{}{}
Extract all the sentences of the document

\end{fulllineitems}


\end{fulllineitems}

\index{Domain (class in theseus)}

\begin{fulllineitems}
\phantomsection\label{theseus:theseus.Domain}\pysiglinewithargsret{\strong{class }\code{theseus.}\bfcode{Domain}}{\emph{label}, \emph{words=}\optional{}}{}
A Domain is a field with a collection of words and a label

Domain words should all be lower capital and without stopwords!

\end{fulllineitems}

\index{Sentence (class in theseus)}

\begin{fulllineitems}
\phantomsection\label{theseus:theseus.Sentence}\pysiglinewithargsret{\strong{class }\code{theseus.}\bfcode{Sentence}}{\emph{text}, \emph{lang='en'}}{}
The Sentence is one of the building blocks of Documents
\index{cleanText() (theseus.Sentence method)}

\begin{fulllineitems}
\phantomsection\label{theseus:theseus.Sentence.cleanText}\pysiglinewithargsret{\bfcode{cleanText}}{}{}~\begin{description}
\item[{Processes the raw text of a sentence:}] \leavevmode\begin{itemize}
\item {} 
creates a \code{cleaned} text without unauthorized letters,

\item {} 
creates a \code{words} list

\item {} 
creates a \code{cleanedWords} list without \textbf{stopwords}

\end{itemize}

\end{description}

\end{fulllineitems}


\end{fulllineitems}

\index{binary() (in module theseus)}

\begin{fulllineitems}
\phantomsection\label{theseus:theseus.binary}\pysiglinewithargsret{\code{theseus.}\bfcode{binary}}{\emph{token}, \emph{doc}}{}
Calculates the Binary existence of a token in a document (doc)

returns 1 if token exists, 0 otherwise
\begin{itemize}
\item {} 
\emph{token} is a string               ex. `word'

\item {} 
\emph{doc} is a \code{list}                 ex. {[}'this' `is' `a' `word' `document'{]}

\end{itemize}

\end{fulllineitems}

\index{cleanString() (in module theseus)}

\begin{fulllineitems}
\phantomsection\label{theseus:theseus.cleanString}\pysiglinewithargsret{\code{theseus.}\bfcode{cleanString}}{\emph{s1}}{}
Cleans strings from unauthorized letters

\end{fulllineitems}

\index{cleanStringNoDel() (in module theseus)}

\begin{fulllineitems}
\phantomsection\label{theseus:theseus.cleanStringNoDel}\pysiglinewithargsret{\code{theseus.}\bfcode{cleanStringNoDel}}{\emph{s1}}{}
Cleans strings from unauthorized letters

\end{fulllineitems}

\index{clusterHist() (in module theseus)}

\begin{fulllineitems}
\phantomsection\label{theseus:theseus.clusterHist}\pysiglinewithargsret{\code{theseus.}\bfcode{clusterHist}}{\emph{clst}}{}
Takes a List of DocNodes and returns an histogram of the most common words

\end{fulllineitems}

\index{dtf() (in module theseus)}

\begin{fulllineitems}
\phantomsection\label{theseus:theseus.dtf}\pysiglinewithargsret{\code{theseus.}\bfcode{dtf}}{\emph{token}, \emph{corpus}}{}
Calculates the fraction of documents of the corpus that have a token
\begin{itemize}
\item {} 
\emph{token} is a string               ex. `word'

\item {} 
\emph{corpus} is a \code{list} of \code{lists}    ex. {[}{[}'this' `is' `a' `word' `document'{]}{[}'document' `two'{]}{]}

\end{itemize}

\end{fulllineitems}

\index{enClean() (in module theseus)}

\begin{fulllineitems}
\phantomsection\label{theseus:theseus.enClean}\pysiglinewithargsret{\code{theseus.}\bfcode{enClean}}{}{}
English Stop Words

\end{fulllineitems}

\index{extractPhrases() (in module theseus)}

\begin{fulllineitems}
\phantomsection\label{theseus:theseus.extractPhrases}\pysiglinewithargsret{\code{theseus.}\bfcode{extractPhrases}}{\emph{s1}}{}
extractPhrases breaks a document into a sequence of phrases.

XXX: We need to deal with numbers...

\end{fulllineitems}

\index{idf() (in module theseus)}

\begin{fulllineitems}
\phantomsection\label{theseus:theseus.idf}\pysiglinewithargsret{\code{theseus.}\bfcode{idf}}{\emph{token}, \emph{corpus}}{}
Calculates the inverse document frequency of a token
\begin{itemize}
\item {} 
\emph{token} is a string               ex. `word'

\item {} 
\emph{corpus} is a \code{list} of \code{lists}    ex. {[}{[}'this' `is' `a' `word' `document'{]}{[}'document' `two'{]}{]}

\end{itemize}

\end{fulllineitems}

\index{jaccard() (in module theseus)}

\begin{fulllineitems}
\phantomsection\label{theseus:theseus.jaccard}\pysiglinewithargsret{\code{theseus.}\bfcode{jaccard}}{\emph{s1}, \emph{s2}}{}
Calculates de jaccard index for two lists

\end{fulllineitems}

\index{logtf() (in module theseus)}

\begin{fulllineitems}
\phantomsection\label{theseus:theseus.logtf}\pysiglinewithargsret{\code{theseus.}\bfcode{logtf}}{\emph{token}, \emph{doc}}{}
Calculates the Log Term Frequency in a certain document (doc)
\begin{itemize}
\item {} 
\emph{token} is a string   ex. `word'

\item {} 
\emph{doc} is a \code{list}     ex. {[}'this' `is' `a' `word' `document'{]}

\end{itemize}

\end{fulllineitems}

\index{logtfidf() (in module theseus)}

\begin{fulllineitems}
\phantomsection\label{theseus:theseus.logtfidf}\pysiglinewithargsret{\code{theseus.}\bfcode{logtfidf}}{\emph{token}, \emph{doc}, \emph{corpus}}{}
Calculates the Log Term Frequency-Inverse Document Frequency of a token
\begin{itemize}
\item {} 
\emph{token} is a string               ex. `word'

\item {} 
\emph{doc} is a \code{list}                 ex. {[}'this' `is' `a' `word' `document'{]}

\item {} 
\emph{corpus} is a \code{list} of \code{lists}    ex. {[}{[}'this' `is' `a' `word' `document'{]}{[}'document' `two'{]}{]}

\end{itemize}

\end{fulllineitems}

\index{normF() (in module theseus)}

\begin{fulllineitems}
\phantomsection\label{theseus:theseus.normF}\pysiglinewithargsret{\code{theseus.}\bfcode{normF}}{\emph{token}, \emph{channel}}{}
Calculates the normalized frequency of a term in a channel of documents

see {\hyperref[theseus:theseus.tfpdf]{\code{theseus.tfpdf()}}}

\end{fulllineitems}

\index{ptClean() (in module theseus)}

\begin{fulllineitems}
\phantomsection\label{theseus:theseus.ptClean}\pysiglinewithargsret{\code{theseus.}\bfcode{ptClean}}{}{}
Portuguese Stop Words

\end{fulllineitems}

\index{tf() (in module theseus)}

\begin{fulllineitems}
\phantomsection\label{theseus:theseus.tf}\pysiglinewithargsret{\code{theseus.}\bfcode{tf}}{\emph{token}, \emph{doc}}{}
Calculates the term frequency in a certain document (doc)
\begin{itemize}
\item {} 
\emph{token} is a string   ex. `word'

\item {} 
\emph{doc} is a \code{list}     ex. {[}'this' `is' `a' `word' `document'{]}

\end{itemize}

\end{fulllineitems}

\index{tfidf() (in module theseus)}

\begin{fulllineitems}
\phantomsection\label{theseus:theseus.tfidf}\pysiglinewithargsret{\code{theseus.}\bfcode{tfidf}}{\emph{token}, \emph{doc}, \emph{corpus}}{}
Calculates the Term Frequency-Inverse Document Frequency of a token
\begin{itemize}
\item {} 
\emph{token} is a string               ex. `word'

\item {} 
\emph{doc} is a \code{list}                 ex. {[}'this' `is' `a' `word' `document'{]}

\item {} 
\emph{corpus} is a \code{list} of \code{lists}    ex. {[}{[}'this' `is' `a' `word' `document'{]}{[}'document' `two'{]}{]}

\end{itemize}

\end{fulllineitems}

\index{tfpdf() (in module theseus)}

\begin{fulllineitems}
\phantomsection\label{theseus:theseus.tfpdf}\pysiglinewithargsret{\code{theseus.}\bfcode{tfpdf}}{\emph{token}, \emph{channels}}{}~\begin{quote}

Calculates the Term Frequency * Proportional Document Frequency (TF*PDF )
\begin{itemize}
\item {} 
\emph{token} is a string

\item {} 
\emph{channels} is a \code{list} of \code{Channel}

\end{itemize}
\end{quote}
\paragraph{References}

\end{fulllineitems}



\subsection{crawler.py (will available in v.0.7)}
\label{theseus:crawler-py-will-available-in-v-0-7}

\subsection{utils.py (will available in v.0.8)}
\label{theseus:utils-py-will-available-in-v-0-8}
See {\hyperref[TODO::doc]{\emph{Roadmap}}} for details


\section{How to}
\label{howto::doc}\label{howto:how-to}
\emph{Some simple examples to get you started using python and Theseus}


\subsection{Process 11 TXT files inside a ``TXT'' folder}
\label{howto:module-eccs10bursaries}\label{howto:process-11-txt-files-inside-a-txt-folder}\index{eccs10bursaries (module)}

\subsubsection{eccs10bursaries.py}
\label{howto:eccs10bursaries-py}
This example will demonstrate the use of Theseus to process a set of texts
that are archived in a folder ./TXT

Text Files are named 01.txt ... 11.txt
\index{main() (in module eccs10bursaries)}

\begin{fulllineitems}
\phantomsection\label{howto:eccs10bursaries.main}\pysiglinewithargsret{\code{eccs10bursaries.}\bfcode{main}}{}{}
This example will demonstrate the use of Theseus to process a set of texts
that are archived in a folder ./TXT

Text Files are named 01.txt ... 11.txt

Check the source code to detailed step by step instructions

\end{fulllineitems}



\section{Frequently Asked Questions (FAQ)}
\label{faq::doc}\label{faq:frequently-asked-questions-faq}
\emph{Common questions and answers for common (sometimes) problems}


\section{Contact The Observatorium}
\label{contact::doc}\label{contact:contact-the-observatorium}
The Observatorium Webstie is at \href{http://www.theobservatorium.eu}{http://www.theobservatorium.eu}

David Rodrigues email is \href{mailto:m4467@iscte.pt}{m4467@iscte.pt}


\section{Roadmap}
\label{TODO::doc}\label{TODO:roadmap}

\subsection{0.8}
\label{TODO:id1}\begin{itemize}
\item {} 
add \code{utils.py} and collect some dispersed scripts into this package.

\item {} \begin{description}
\item[{solve abrveviation problems in the identification of phrases}] \leavevmode
ex. ``His name is D. Rodrigues and he his a scientist''. The dot after D will break a sentence.
So one needs to be awere of this.
Another problem is that of the use of hiffens. a ``pre-conference'' should be treated as 1 word and not as two.
This things have to be processed at the Document level before breaking the Document into Sentences

\end{description}

\end{itemize}


\subsection{0.7}
\label{TODO:id2}\begin{itemize}
\item {} 
add \code{fr} stop words

\item {} 
add \code{es} stop words

\item {} 
rename \textbf{theseus} module to \textbf{processor} and incorporate \code{crawler.py} code into \emph{Theseus} as \textbf{crawler} module

\item {} 
\textbf{Documentation} Write what is Thesues section of this documentation.

\end{itemize}


\subsection{0.6 \textbf{Present Version}}
\label{TODO:present-version}\begin{itemize}
\item {} 
implement {\hyperref[theseus:theseus.tfpdf]{\code{theseus.tfpdf()}}} method \textbf{{[}Done{]}}

\item {} 
test {\hyperref[theseus:theseus.tfpdf]{\code{theseus.tfpdf()}}} with text from ECCS`10 Bursaries \textbf{{[}Done{]}}

\item {} 
\textbf{Documentation} Write the ECCS`10 Bursaries text as an example of usage. \textbf{{[}Done{]}}

\end{itemize}


\subsection{0.5.1}
\label{TODO:id3}

\chapter{Indices and tables}
\label{index:indices-and-tables}\begin{itemize}
\item {} 
\emph{genindex}

\item {} 
\emph{modindex}

\item {} 
\emph{search}

\end{itemize}


\renewcommand{\indexname}{Python Module Index}
\begin{theindex}
\def\bigletter#1{{\Large\sffamily#1}\nopagebreak\vspace{1mm}}
\bigletter{e}
\item {\texttt{eccs10bursaries}}, \pageref{howto:module-eccs10bursaries}
\indexspace
\bigletter{t}
\item {\texttt{theseus}}, \pageref{theseus:module-theseus}
\end{theindex}

\renewcommand{\indexname}{Index}
\printindex
\end{document}
